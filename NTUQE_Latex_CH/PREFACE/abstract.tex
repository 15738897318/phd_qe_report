% ----------------------------------------------------------------------------
% Abstract
% ----------------------------------------------------------------------------
\indent Signal processing have been widely used in many computational based applications such as in the areas of image and video processing, smart sensors network, and other real-time processing embedded system. Many signal processing systems also involve signal sampling and conversion, where one of the most crucial limitations lies in the sampling theory (the Nyquist sampling rate or Shannon theory.) 

Due to limited data processing and storing capability found in many embedded systems, excessive data samples will negatively affect their performance and power efficiency, and hence forms the bottleneck of the entire digital systems.

Compressive sensing (or compressed sensing) (CS) technique, is a relatively novel paradigm for signal acquisition that has great potential to address the problems mentioned above. The CS theory has been shown to be suitable for numerous computer science and electronic engineering based applications, where it is possible to overcome the restriction of the conventional sampling theory.

The main focus of the research work is to further extend the CS based signal acquisition techniques for CS based data processing, particular in the wireless based applications. Using this approach, the proposed directions of the research will encompass various areas such as data compression, data acquisition, data storage, data transmission, optimal recovery and processing. Specifically, the focus will be for wideband signal processing based application since the requirement of sampling wideband spectrum are much more demanding than narrow band signal. 

This report presents the research motivation that are targeted for areas involving extremely high frequency acquisition and processing scenarios, such as the ultra wideband (UWB) positioning applications and cognitive spectrum sensing (CSS) systems.

