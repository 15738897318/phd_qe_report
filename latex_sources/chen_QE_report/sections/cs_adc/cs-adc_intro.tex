
According to the Nyquist Theorem, we reconstruct signal from the uniformly sampled data if the acquisition rate is at least 2 times higher than the highest frequency component of an original signal\cite{baraniuk2007compressive,candes2008introduction,davenport2011introduction}. However, this traditional method becomes cumbersome in case of sampling wide-band signals which contains a relatively high frequency component. Therefore, when the requirement of sampled bandwidth increases, power consumption and requirements of data transmission and storage increase significantly, consequently becomes a bottleneck of the entire signal processing systems.

To solve this problem, some compressive methods such as wavelet compression develop the sparsity\cite{starck2010sparse} in signals – This new approach first projects the input signal to a novel orthogonal basis in order to make the representation of the original signal contains limited non-zero coefficients. Then the system preserves the main information of the original signal via keeping only largest K non-zero coefficients and throw away the rest of them. However, this compression approach relies on a prior knowledge of coefficient locations in sparse representation, which requires sufficient samples in time domain for analysis, and hence not approximate for signal acquisition system aiming at reducing the sampling rate\cite{han2013compressive}. Based on sparsity but different from traditional sparse compression (CS), compressive sensing samples signals without knowing the coefficient locations -- CS can apply a randomly sampling measurement to satisfy the RIP\cite{baraniuk2007compressive} such that it successfully reconstruct sparse information of the input signal with high probability, especially only requiring $O(K\log(N/K))$ samples (where K is the sparsity of the signal) which is much less than the requirement of $N$ samples based on the Nyquist Theory\cite{foucart2013mathematical}.    Therefore, CS based acquisition systems can significantly reduce the scale of measurements and accordingly improve the power efficiency and performance of data transmission and storage.   

As input devices of data acquisition system, recent analog to digital converters (ADCs) is of significance that many recent ADCs implement and combine the CS theory in their architectures, and successfully benefit power consumption from a relatively lower sampling rate provided by CS based novel architectures. However, the CS based ADCs still meet several problems that restrict them to gain higher performance theoretically and practically, including how to efficiently constructing applicable sensing measurement or choosing suitable dictionaries for hardware design\cite{davenport2012signal,davenport2012compressive}, and how to design a faster reconstruction or signal processing algorithms but not lose the stability and robustness\cite{mishali2011xamplingsignal,davenport2010signal} etc. (\cite{mishali2011xamplingsignal} adopts three metrics for the choice of analog compression: robustness to model mismatch, required hardware accuracy and software complexities) Nowadays some attractive progress for these problems are achieved, embedded in some popular CS-ADC architectures like random demodulators\cite{kirolos2006analog}, modulated wideband converters\cite{mishali2010theory}, Xampling\cite{mishali2012xampling}etc.
%chen% 
%For instance, the limitation of how to construct an appropriate pseudo-random measurement in hardware design to replace the real random sampling matrix in CS theory restricts the stability and robustness of CS based ADC\cite{baraniuk2007compressive}. Another constraint is about the real-time processing ability\cite{baraniuk2007compressive}. Since the compressed data via CS theory cannot easily recovered or filtered from observations compared to the traditional sampling method, CS reconstruction and CS processing are likely more time consuming and complex, and it is crucial problem in handhold and battery devices\cite{baraniuk2007compressive}.
%chen% 

This article first reviews the basic paradigm of the CS theory for signal acquisition and reconstruction in Section2. Then Section3 focuses on the sparse reconstruction algorithms. The following Section4 introduces prevailing applications of CS based ADCs for signal processing, and present discussion based on architectures and performance. Finally, the summary of prevailing CS-ADCs architectures and future work will be included in Section5.
