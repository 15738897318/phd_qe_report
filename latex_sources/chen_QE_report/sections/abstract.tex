%abstract.tex
Compressive sensing (CS) is a new signal-processing paradigm that enable reconstruction of sparse signals that are sampled at sampling rates much lower than the conventional Nyquist sampling system while at the same time, enable better SNR. As such, it has great potential for high data rate applications, and has attracted significant attention from researchers and engineers working in various areas such as wireless communication, multimedia processing interweaving signal processing, information theory, communications and networking. This report presents a summary of the study performed on the topic of compressive sensing, including background and basics of compressive sensing theory, an understanding of its benefits and limitations. It also contains my recent work progress in applying CS for analog-to-digital converters (ADCs) and for ultra-wideband (UWB) positioning system. It then outline potential application areas of CS in wireless networks, which is tentatively identified as the direction for my future work.

