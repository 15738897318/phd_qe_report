\subsection{Compressed Sensing channel estimation}
% to rewrtie
High-rate data communication over a multipath wireless channel often requires that the channel response be known at the receiver. Training-based methods, which probe the channel in time, frequency, and space with known signals and reconstruct the channel response from the output signals are most commonly used to accomplish this task. However, real-world experience and growing experimental evidence suggest that many wireless channels requires large signal space dimension due to large bandwidth or large number of antennas. For instance, wide band channel estimation need high frequency sampling at receivers which places heavy burden on data storage and power cost due to extremely high sampling rate. For such cases, compressed sensing based approach has high potential to overcome the limitation by using far less energy and, in many instances, smaller latency and bandwidth than that dictated by the traditional least-squares-based training methods\cite{bajwa2010compressed}. This is being investigated in our recent work on UWB positioning system, where the use of CS-based channel estimation shows potential to provide better performance for the system.

\subsection{Compressed Sensing Positioning}
% to rewrtie
In typical positioning system, , the target needs to send signals to the base stations or receive signals from base stations in order to determine the target’s position, such that the position of the target can be determined based on time of arrival (TOA), angle of arrival (AOA) algorithms etc. There are two general ways how compressive sensing can be useful for such applications\cite{han2013compressive}. The first one, which is considered as more traditional positioning technique, is to apply CS to enhance the signal’s SNR for more accurate positioning estimation,which also improve the power efficiency. The second way is to establish a linear mapping from the information indicating the positions to the observations, and then calculate the positioning using CS based algorithm to solve the unknowns used in the mapping. Our current investigation related to CS-based UWB positioning system for indoor environment, where CS-based receivers or transmitter using sub-Nyquist sampling rate to detect high frequency UWB signals, has shown that it is effective on increasing signal SNR and position accuracy.

\subsection{Compressed Sensing Cognitive Radio}
% to rewrtie
The increasing demand for higher data rates in wireless communications in the face of limited or under utilized spectral resources has motivated the introduction of cognitive radio. Traditionally, licensed spectrum is allocated over relatively long time period and is intended to be used only by licensees. Various measurements of spectrum utilization have shown substantial unused resources in frequency, time, and space\cite{axell2012spectrum}. The concept behind cognitive radio is to exploit these underutilized spectral resources by reusing unused spectrum in an opportunistic manner. In many cognitive radio applications, a wide band of spectrum must be sensed, which requires high sampling rates and thus high power operating in the A/D converters. One solution to this problem is to divide the wideband channel into multiple parallel narrowband channels and to jointly sense transmission opportunities on those channels. This technique is called multiband sensing. Another approach argues that the interference from the primary users can often be interpreted as being sparse in some particular domain, such as in the spectrum. In that case, compressive sensing can be used to lower the burden on the A/D converters. In addition, cooperative versions of compressive wideband sensing have also been developed \cite{tian2008compressed,wang2009distributed}. Here, individual radios can make a local decision about the presence or absence of a primary user, and these results can then be used in a centralized or decentralized manner.

\subsection{Compressed Sensing Multiple Access}
% to rewrtie
In wireless communications, an important task is the multiple access that resolves the collision of the signals sent from multiple users. Traditional studies assume that all users are active and thus the technique of multiuser detection can be applied. However, in many practical systems such as wireless sensor networks, only a random and small fraction of users send signals simultaneously.. Hence, based on the observation that in some cases the useful information of active multiusers is sparse, multiple access problem can be more efficiently resolved by using a CS approach. In particular, the feature of discrete unknowns can be incorporated into the reconstruction algorithm. Furthermore, the CS-based multiple access scheme can also be integrated with the channel coding to further provide better performance of the system.
