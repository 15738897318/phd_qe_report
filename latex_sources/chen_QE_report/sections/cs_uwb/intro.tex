%Introduction
Ultra-wide band (UWB) communication is widely used in wireless communication and associated with features as extreme wide transmission bandwidth, low-power consumption, shared spectrum resources in wide ranges etc \cite{paredes2007ultra}. Among all applicable areas, UWB based accurate positioning and tracking in short range communication become popular since it provide resilient to multipath fading in hostile environment and outstanding robustness even in low signal-to-noise (SNR) situations \cite{cassioli2002ultra}. In addition, the requirements of high data rate and limitations of battery supply lead the impulse-radio ultra-wide band (IR-UWB) to become a suitable communication technique in short range high data rate communication. As a result, applications based short-distance wireless sensor networks (WSNs) where large numbers of portable instrument, are widely deployed such as indoor positioning, surveillance, home automation, etc. 

However, high data rate transmission puts huge pressure on signal detection at ADCs at receivers, which indicates the sampling rate becomes a main bottleneck to the IR-UWB system. This paper focuses on the IR-UWB indoor positioning, aims at solving the bottleneck of sampling rate by using a recent novel technique termed compressed sensing (CS). Compressed sensing is a novel paradigm which applies randomly sampling and sparse reconstruction, which enables a possible reconstruction strategy for sparse signals from a relatively small group of random measurements. It indicates that CS based IR-UWB systems can possibly detect high frequency sparse signals under a sub-Nyquist sampling rates that far below the Nyquist rate (twice the IR-UWB bandwidth). This result is advantageous which releases the bottleneck of the large bandwidth constraints on ADC at UWB receivers, and consequently reduces storage limits and improves energy efficiency in IR-UWB positioning systems.

Recently some researches manage to embed CS reconstruction algorithms, e.g. revised orthogonal matching pursuit, at UWB receivers. These algorithms successfully improve the SNR of the received signal before they are sent to the stages of time of arrival (TOA) based positioning algorithm. Consequently, this method increases the performance of entire positioning accuracy \cite{banitalebi2014compressive}. For hardware implementation of the new CS based UWB receivers, most of them apply the hardware structure termed the random demodulator (RD) \cite{kirolos2006analog}, and consequently this new CS-UWB receiver successfully reduces the sampling rate significantly compared to the Nyquist rate \cite{yang2011compressive}.

On the other hand, some researches embed the CS technique mainly at UWB transmitters. They develop a waveform-based precoding transmitter, in order to fulfill a random projection of the UWB generated pulses \cite{zhang2009compressed}. Followed by sub-Nyquist sampling ADCs at receivers, the sampled signals are sent to TOA based algorithm for calculating the location of the UWB transmitter. Simulation results show that the new CS-UWB transmitter manage to significantly decrease the sampling rate of receivers successfully improves accuracy of traditional UWB positioning system. 

However, both CS-UWB receivers and CS-UWB transmitters suffer from high-data rate random mixing operation, where PN sequence at mixers is required to reach the extremely high Nyquist rate ( e.g. beyond 10 GHz). This requirement generates heavy burden on bandwidth of hardware mixers and additionally increases a high frequency noise. To solve this problem, this paper propose an advanced low-rate CS-UWB positioning system: For the CS-UWB transmitter, it implements a relatively low-rate random projection matrix to slow down the mixing rate. For CS-UWB receivers, they sacrifices a small degree of compression ratio to keep equivalent performance as those positioning system using traditional CS-UWB transmitters. As a result, the trade-off between the random mixing rate and the sub-Nyquist sampling rate makes our system to become a more energy balanced, and consequently gains a better performance in entire power consumption and energy efficiency.  

This article first theoretically reviews compressed sensing paradigm and UWB indoor positioning system in Section2. Next, Section3 discusses popular CS based UWB positioning and proposes the new structure of CS-UWB positioning. The following Section4 and Section5 presents simulation results and conclusion respectively.    

